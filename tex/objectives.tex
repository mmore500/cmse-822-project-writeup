\section{Objectives}

The objective of this project is to compare the parallel performance of alternate frameworks for the DISHTINY model: MPI and Charm++.

Charm++ requires the user to abstract their computation across of a set of Chare objects, written much like C++ objects.
Chare objects, written much like C++ objects, interact by calling each other's invocation methods.
These calls return immediately and, under the hood, the arguments to the call are wrapped into a message and then delivered to the Chare object.
These objects are adaptively scheduled across available computational resources by Charm++'s runtime system.
Charm++ is developed and maintained by the Parallel Programming Laboratory at the University of Illinois \cite{kale1993charm+}.

MPI defines syntax and semantics for the building blocks of parallel programs \cite{Forum:1994:MMI:898758}.
MPI executes the same code across many processors, with instances communicating through message-passing or remote memory access.
Unlike Charm++, explicit decisions about precisely how MPI ranks will run and interact must be made by the programmer before compile time.
MPI is a well-established standard with several mature implementations.

Comparing Charm++ with MPI allows exploration of the performance and behavior of synchronous versus asynchronous approaches to parallelizing DISHTNINY.
One one end of the spectrum, individual toroidal grid tiles might proceed in lockstep update by update, with each tile waiting for all others to complete the current update before proceeding to the next.
We implement this synchronous approach with MPI.
On the other end of the spectrum, individual toroidal grid tiles might be allowed to proceed asynchronously with resource waves with any cell-to-cell communication proceeding as quickly (and potentially unevenly) as runtime conditions allow.
We implement this asynchronous approach with Charm++.

For tractability, this work uses a stripped-down version of the DISHTINY model, simulating only resource-wave events and resource accumulation.
In future work, we are eager to realize actual evolution with cell-like organisms capable of arbitrary computation via genetic programming in order to pursue more open-ended evolutionary experiments \cite{lalejini2018evolving}.
Our scaling study will point the way forward for future development of parallel DISHTINY and the implementations put together for the project will provide a jumping-off point for that development.
