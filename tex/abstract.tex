\begin{abstract}

The emergence of new replicating entities from the union of simpler entities represent some of the most profound events in natural evolutionary history.
Such transitions in individuality are essential to the evolution of the most complex forms of life.
As such, understanding these transitions is critical to building artificial systems capable of open-ended evolution.
Alas, these transitions are challenging to induce or detect, even with computational organisms.
In previous work \cite{moreno2018toward}, we introduced the DISHTINY (DIStributed Hierarchical Transitions in IndividualitY) model, which provides simple cell-like organisms with the ability and incentive to unite into new individuals in a manner that can continue to scale to subsequent transitions.
The system is designed to encourage these transitions so that they can be studied:
organisms that coordinate spatiotemporally can maximize the rate of resource harvest, which is closely linked to their reproductive ability.
The objective of this project is to explore the performance and behavior of synchronous and asynchronous to scaling the DISHTINY model with MPI.
For tractability, we intend to work with ``a model of the model'' (e.g., placeholder computations and messages designed to resemble the computational activity and messaging patterns of our system without actually implementing the content of those computations and messages) instead of the full DISHTINY model.

\end{abstract}
