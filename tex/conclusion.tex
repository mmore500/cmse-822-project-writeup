\section{Conclusion}

The goal of this project is to compare asynchronous and synchronous approaches to parallelizing digital evolution experiments on fraternal transitions of individuality.
The asynchronous approach was implemented with with Charm++ and the synchronous approach was implemented with MPI.

Charm++ provides an appealing message-driven paradigm would align nicely with existing software in our lab group designed for event-driven evolutionary computation \cite{lalejini2018evolving}.
Exploiting Charm++'s object-oriented framework and simple method invocation to message syntax would allow for intuitive and rapid re-use of this existing software.

On the grounds of performance, though, MPI proves the better choice.
The strong scaling of our MPI implementation is much closer to ideal than that of Charm++ implementation, particularly as the first 64 CPUs were added.
In addition, although MPI implementation took a performance hit moving from single-node to multi-node execution, the near-identical times to solution when run on 64 and 256 cores in our weak scaling study suggests that we might be able to perform further scaling past 256 processors efficiently.

In addition, MPI allows for more explicit control of important aspects of parallel programming.
For example, MPI allows the programmer to specify that ranks interact in a grid topology so that the placement.
No such control is possible in Charm++; the programmer must instead rely upon the adaptive runtime system to pick up on communication patterns and migrate Chares appropriately
Working with MPI will allow us to much more tightly fine-tune our software.
It's faster and allows better control.

Finally, because MPI is much more ubiquitous than Charm++ it benefits from better support and integration with other software, such as parallel I/O with HDF5 or checkpoint-restart with DMCTP.

Although Charm++ offers an appealing object-oriented and event-driven software development framework, MPI is the obvious choice moving forward.
We hope that exploiting parallel computing power will allow us to perform much larger-scale evolutionary transitions of individuality experiments in order to better understand these important evolutionary events.
