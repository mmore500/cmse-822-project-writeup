\section{Conclusion}

Using simple organisms that evolve parameters for a set of manually-designed strategies, we have demonstrated that DISHTINY selects for genotypes that exhibit high-level individuality.
We observed a spectrum of first- and second- level individuality among evolutionary outcomes.
Specifically, we observed
\begin{enumerate}
  \item reproductive division of labor among members of the same channel (i.e., individuals enveloped in a same-channel signaling network ceded reproduction to those at the periphery),
  \item cooperation between members of the same channel (i.e., pooling of resource on same-channel signaling networks),
  \item reproductive bottlenecking (i.e., groups of cells sharing a channel ID descend from a single originator of that channel ID), and
  \item suppression of somatic mutation via apoptosis coincident with second-level individuality.
\end{enumerate}

Competition experiments revealed that second-level individuals usually outcompete lower-level individuals.
The magnitude of resource endowment for propagules was also correlated with second-level individuality.

Although shifts in individuality to level-one and level-two signaling networks were both observed, the question of whether these transitions were truly hierarchical in nature is debatable.
That is, it is not clear whether level-one individuality was to some extent preserved in or necessary for the emergence of level-two individuality.
Given the nature of the manually-designed strategies for resource-pooling and reproductive division of labor, level-two resource pooling and division of labor could readily leapfrog over level-one resource pooling and division of labor and, in many ways, seemed to completely supersede those level-one efforts.

We believe that this is a shortcoming of the manual design of behaviors for which simple cell-like organisms evolved parameters, not the DISHTINY platform itself.
We have nevertheless demonstrated that DISHTINY ultimately selects for high-level individuality.
We are eager to work with more sophisticated cell-like organisms capable of arbitrary computation via genetic programming in order to pursue more open-ended evolutionary experiments.
We will also test the implications of relaxing current arbitrary restrictions that artificially promote transitions, such as the hierarchical nesting of same-channel signaling networks and the explicitly-defined signaling networks themselves, leaving these details to evolution to figure out.
Further work will provide valuable insight into scientific questions relating to major evolutionary transitions such as the role of pre-existing phenotypic plasticity \citep{clune2007investigating, lalejini2016evolutionary}, pre-existing environmental interactions, pre-existing reproductive division of labor, and how transitions relate to increases in organizational \citep{goldsby2012task}, structural, and functional \citep{goldsby2014evolutionary} complexity.

We believe that such an approach also provides a unique opportunity to fundamentally advance Artificial life with respect to open-ended evolution.
Fundamental to this goal is scale.
The DISHTINY platform trivially scales to select for an arbitrary number of hierarchical levels of individuality (not just the two hierarchical levels explored in these experiments).
Importantly, the platform is implemented in a decentralized manner and can comfortably scale as additional computing resources are provided.
Parallel computing is widely exploited in evolutionary computing, where subpopulations are farmed out for periods of isolated evolution or single genotypes are farmed out for fitness evaluation
\citep{lin1994coarse, real17a}.
DISHTINY presents a more fundamental parallelization potential: principled parallelization of the evolving individual phenotype at arbitrary scale (i.e., a high-level individual as a large collection of individual cells on the toroidal grid).
Such parallelization will be key to realizing evolving computational systems with scale --- and, perhaps, complexity --- approaching biological systems.
