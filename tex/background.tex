\section{Background}

Artificial life researchers design systems that exhibit properties of biological life in order to better understand their dynamics and, often, to apply these principles toward engineering applications such as artificial intelligence \cite{bedau2003artificial}.
Evolutionary transitions in individuality, which are key to the complexification and diversification of biological life \cite{smith1997major}, have been highlighted as key research targets with respect to the question of open-ended evolution \cite{ray1996evolving, banzhaf2016defining}.
In an evolutionary transition of individuality, a new, more complex replicating entity is derived from the combination of cooperating replicating entities that have irrevocably entwined their long-term fates \cite{west2015major}.
In particular, we focus on fraternal transition in individuality, events where closely-related kin come together or stay together to form a higher-level organism \cite{queller1997cooperators}.
Eusocial insect colonies and multicellular organisms exemplify this phenomenon \cite{smith1997major}.
